\documentclass[12pt]{article}

\usepackage[nottoc]{tocbibind}
\usepackage[colorlinks=true,
            linkcolor=blue,
            urlcolor=blue,
            citecolor=black]{hyperref}

\begin{document}

\begin{center}
\Large \textbf{Project Plan for Degree Projects} \\
\large \textbf{Department of Computer Science}
\end{center}

\section*{General Information}
\begin{tabular} {|p{3.8cm}|p{9cm}|} \hline
Title: & Reel-time communication using a peer-to-peer network in the web \\ \hline
External company: & Name of the company (if you do your degree project at an external company) \\ \hline
\end{tabular}

\section*{Persons involved}
\begin{tabular} {|p{2.2cm}|p{4.7cm}|p{5.47cm}|} \hline
Student 1: & Henry Pap & hp222fq@student.lnu.se \\ \hline
Student 2: &  &  \\ \hline
\end{tabular}
\\ \vspace*{0.2cm} \\
\begin{tabular} {|p{3.9cm}|p{8.9cm}|} \hline
Supervisor: & Mauro Caporuscio \textit{mauro.caporuscio@lnu.se} \\ \hline
External supervisor: & Your supervisor at the company (if you do your degree project at an external company) \\ \hline
\end{tabular}

\section*{Background}
Describe the background to the problem area of your degree project. Max half page.

\section*{Problem formulation}
Briefly describe the problem you plan to investigate, its limitations, and what results you expect. You can read about suitable problems \href{https://coursepress.lnu.se/subject/thesis-projects/problem/}{here}.

\section*{Motivation}
Briefly describe why your problem is interesting for science, society or industry. Use references in IEEE format\cite{Chalmers:2013,Darwin:1859}.

\section*{Objectives}

\begin{tabular} {|p{1.2cm}|p{11.6cm}|} \hline
\textbf{O1} & How to achieve scalability for real-time peer-to-peer network in the web. \\ \hline
\textbf{O2} & Determine the most suitable peer-to-peer topology. \\ \hline
\textbf{O3} & Achieving low latency for real time multiplayer games. \\ \hline
\end{tabular}

\section*{Method}
A systematic literature review will be conducted in order to gather background 
knowledge about peer to peer network topologies. For determine the most suitable 
peer-to-peer network topology a comparative study has to be done thus ranking the 
topologies. And by experimentation on them with their corresponding settings, 
achieving the best topology based on both low latency and scalability. 


\noindent\newline From the experiment-results, picking the most suitable topology 
and implementing it using webRTC API and then verify and validate it.

\subsection*{Methods used}
\begin{enumerate}
  \item Systematic Literature Review
  \item Comparative study
  \item Controlled Experiment
  \item Verification and Validation 
\end{enumerate}

\section*{Time plan}
\begin{tabular} {|p{2.6cm}|p{10.2cm}|} \hline
\textbf{Date} & \textbf{Milestone} \\ \hline
2019-02-15 & Systematic Literature Review for background knowledge (SLR)\\ \hline
2019-02-20 & Degree project plan finished \\ \hline
2019-03-08 & Reading and gather the information generated from the SLR \\ \hline
2019-03-10 & selecting 5 peer-to-peer topologies \\ \hline
2019-03-17 & Writing the peer-to-peer background part based on the selected topologies \\ \hline
2019-03-20 & Exploring the webRTC API by both reading the specs and testing with simple implementations  \\ \hline
2019-03-23 & Finishing the background with webRTC \\ \hline
2019-03-25 & Writing the motivation and aim (re-using this document with some retouching) \\ \hline
2019-03-27 & Starting with the (method?) and result \\ \hline
2019-04-05 & Experiment with the different topologies to achieve low latency and scalability \\ \hline
2019-04-15 & Implement the most suitable topology into web environment using the webRTC API \\ \hline
2019-04-20 & Verifying and validating the implementation  \\ \hline
2019-04-22 & Discussing the results based on the evaluation  \\ \hline
2019-04-23 & Discuss the picking of the 5 topologies \\ \hline
2019-04-27 & Finish the conclusion \\ \hline
2019-05-01 & Finalizing the thesis  \\ \hline
2019-05-03 & Preparing the presentation \\ \hline
\end{tabular}

\pagebreak
\bibliographystyle{IEEEtran}
\bibliography{./references}

\end{document}}