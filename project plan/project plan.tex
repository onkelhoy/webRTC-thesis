\documentclass[12pt]{article}

\usepackage[nottoc]{tocbibind}
\usepackage[colorlinks=true,
            linkcolor=blue,
            urlcolor=blue,
            citecolor=black]{hyperref}

\begin{document}

\begin{center}
\Large \textbf{Project Plan for Degree Projects} \\
\large \textbf{Department of Computer Science}
\end{center}

\section*{General Information}
\begin{tabular} {|p{3.8cm}|p{9cm}|} \hline
Title: & Reel-time communication using a peer-to-peer network in the web \\ \hline
External company: & Name of the company (if you do your degree project at an external company) \\ \hline
\end{tabular}

\section*{Persons involved}
\begin{tabular} {|p{2.2cm}|p{4.7cm}|p{5.47cm}|} \hline
Student 1: & Henry Pap & hp222fq@student.lnu.se \\ \hline
Student 2: &  &  \\ \hline
\end{tabular}
\\ \vspace*{0.2cm} \\
\begin{tabular} {|p{3.9cm}|p{8.9cm}|} \hline
Supervisor: & Mauro Caporuscio \textit{mauro.caporuscio@lnu.se} \\ \hline
External supervisor: & Your supervisor at the company (if you do your degree project at an external company) \\ \hline
\end{tabular}

\section*{Background}
Real-time communication (RTC) \cite{Techopedia:RTCdefinition} is a term describing live communication between two or more mediums
with no or very low response time. In the context of networking this means distributed clients (geographically) 
can communicate with each other on a low latency. Today we either use client-server or peer-to-peer architectures
that both come with their own pros and cons. 
Peer-to-peer (P2P) systems in the accepted definition by Theotokis et al. \cite{Androutsellis-Theotokis:2004:SPC:1041680.1041681} are pure and loose P2P systems, where pure refers to total
distribution meaning no use of central servers and loose being something in between of pure P2P and client-server 
without relying to much on centralized servers. 
\pagebreak 

\noindent Web technology offers three main methods for communication (native to the web-browsers):
\begin{enumerate}
  \item HTTP 
  \item WebSockets 
  \item webRTC 
\end{enumerate}

But only WebSockets and webRTC is considered fast enough for RTC, WebSockets
uses what is called a push and receive service meaning that a server also can 
send messages to clients. But only the "newly" API webRTC supports peer-to-peer 
communication as it resolves finding the direct communication-link between two peers. 
Using the web browser comes with drawbacks, it can not handle to much peers and thus the 
recommended number of connected peers to one client is 6. To overcome this issue a specific 
P2P topology is required, by smart routing and use of super-nodes (client that acts like servers)
an virtual network will then be formed.


\section*{Problem formulation}
Peer-to-peer RTC networks is a challenging problem on itself that has been solved multiple times, 
but doing this entirely in the web is not very common. This thesis will focus on multiplayer game networking 
and trying to extend the limit of peer connections in the web using the webRTC API. By investigating and 
using different peer-to-peer topologies that helps extend the peer-connections of a web browser while still 
keeping a low latency potential solutions for real-time multiplayer game will be evaluated. 

\section*{Motivation}
As mentioned in the problem formulation, peer-to-peer real time communication using the web is not something that 
has been solved on a higher level (i.e. having scalability and low latency). From a economical view point this 
helps the game creator to not spend money on a expensive server that handles multiple connections. From a scientific 
point of view, as more and more application is integrated into the web browsers due to its broad availability to almost all 
platforms we need better ways for real-time communication using only the web browser as a toolbox. 

\section*{Objectives}

\begin{tabular} {|p{1.2cm}|p{11.6cm}|} \hline
\textbf{O1} & How to achieve scalability for real-time peer-to-peer network in the web. \\ \hline
\textbf{O2} & Determine the most suitable peer-to-peer topology. \\ \hline
\textbf{O3} & Achieving low latency for real time multiplayer games. \\ \hline
\end{tabular}

\section*{Method}
A systematic literature review will be conducted in order to gather background 
knowledge about peer to peer network topologies. For determine the most suitable 
peer-to-peer network topology a comparative study has to be done thus ranking the 
topologies. And by experimentation on them with their corresponding settings, 
achieving the best topology based on both low latency and scalability. 


\noindent\newline From the experiment-results, picking the most suitable topology 
and implementing it using webRTC API and then verify and validate it.

\subsection*{Methods used}
\begin{enumerate}
  \item Systematic Literature Review
  \item Comparative study
  \item Controlled Experiment
  \item Verification and Validation 
\end{enumerate}

\section*{Time plan}
\begin{tabular} {|p{2.6cm}|p{10.2cm}|} \hline
\textbf{Date} & \textbf{Milestone} \\ \hline
2019-02-15 & Systematic Literature Review for background knowledge (SLR)\\ \hline
2019-02-19 & Degree project plan finished \\ \hline
2019-03-08 & Reading and gather the information generated from the SLR \\ \hline
2019-03-10 & selecting 5 peer-to-peer topologies \\ \hline
2019-03-17 & Writing the peer-to-peer background part based on the selected topologies \\ \hline
2019-03-20 & Exploring the webRTC API by both reading the specs and testing with simple implementations  \\ \hline
2019-03-23 & Finishing the background with webRTC \\ \hline
2019-03-25 & Writing the motivation and aim (re-using this document with some retouching) \\ \hline
2019-03-27 & Starting with the (method?) and result \\ \hline
2019-04-05 & Experiment with the different topologies to achieve low latency and scalability \\ \hline
2019-04-15 & Implement the most suitable topology into web environment using the webRTC API \\ \hline
2019-04-20 & Verifying and validating the implementation  \\ \hline
2019-04-22 & Discussing the results based on the evaluation  \\ \hline
2019-04-23 & Discuss the picking of the 5 topologies \\ \hline
2019-04-27 & Finish the conclusion \\ \hline
2019-05-01 & Finalizing the thesis  \\ \hline
2019-05-03 & Preparing the presentation \\ \hline
\end{tabular}

\pagebreak
\bibliographystyle{IEEEtran}
\bibliography{./references}

\end{document}}