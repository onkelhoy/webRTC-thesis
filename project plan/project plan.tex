\documentclass[12pt]{article}

\usepackage[nottoc]{tocbibind}
\usepackage[colorlinks=true,
            linkcolor=blue,
            urlcolor=blue,
            citecolor=black]{hyperref}

\begin{document}

\begin{center}
\Large \textbf{Project Plan for Degree Projects} \\
\large \textbf{Department of Computer Science}
\end{center}

\section*{General Information}
\begin{tabular} {|p{3.8cm}|p{9cm}|} \hline
Title: & Reel-time communication using a peer-to-peer network in the web \\ \hline
External company: & Name of the company (if you do your degree project at an external company) \\ \hline
\end{tabular}

\section*{Persons involved}
\begin{tabular} {|p{2.2cm}|p{4.7cm}|p{5.47cm}|} \hline
Student 1: & Henry Pap & hp222fq@student.lnu.se \\ \hline
Student 2: &  &  \\ \hline
\end{tabular}
\\ \vspace*{0.2cm} \\
\begin{tabular} {|p{3.9cm}|p{8.9cm}|} \hline
Supervisor: & Mauro Caporuscio \textit{mauro.caporuscio@lnu.se} \\ \hline
External supervisor: & Your supervisor at the company (if you do your degree project at an external company) \\ \hline
\end{tabular}

\pagebreak

\section*{Background}
Real-time communication (RTC) \cite{Techopedia:RTCdefinition} is a term describing live communication between two or more mediums
with no or very low response time. In the context of networking this means distributed clients (geographically) 
can communicate with each other on a low latency. Today we either use client-server or peer-to-peer architectures
that both come with their own pros and cons. 
Peer-to-peer (P2P) systems in the accepted definition by Theotokis et al. \cite{Androutsellis-Theotokis:2004:SPC:1041680.1041681} are pure and loose P2P systems, where pure refers to total
distribution meaning no use of central servers and loose being something in between of pure P2P and client-server 
without relying to much on centralized servers. 

\noindent Web technology offers three main methods for communication (native to the web-browsers):
\begin{enumerate}
  \item HTTP 
  \item WebSockets 
  \item webRTC 
\end{enumerate}

But only WebSockets and webRTC is considered fast enough for RTC, 
WebSockets uses what is called a push and receive service meaning 
that a server also can send messages to clients. But only the "newly" 
API webRTC supports peer-to-peer communication as it resolves finding 
the direct communication-link between two peers. Using the web browser 
comes with drawbacks, it can not handle to much peers and thus the 
recommended number of connected peers to one client is 6. To overcome 
this issue a specific P2P topology is required, by smart routing and 
use of super-nodes (client that acts like servers)an virtual network 
will then be formed.

\pagebreak

\section*{Problem formulation}
Peer-to-peer RTC networks is a challenging problem on itself that has 
been solved multiple times, but doing this entirely in the web is not 
very common. This thesis will focus on multiplayer game networking 
using the native browser-built in webRTC API. The webRTC API has limits 
of connections since it is running directly in the browser.The focus for 
this thesis will thus be to extend the small limit webRTC has while still 
keeping a low latency as this is an very important aspect of real-time 
multiplayer games. The peer-to-peer network must also consider synchronous 
playback that is everyone in the network should get the same feed of data 
at an considerable same time (overall same latency over the whole network). 
Dropout of nodes is another important issue that peer-to-peer brings as 
the dropout peer could be a potential gateway for other peers in the network. 
To sum the problem formulation up, in order for the webRTC API to be a
potential solution for online real-time communication for games an topology
must be carefully picked and finalized.

and extend the webRTC peer-connection limit by using a peer-to-peer 
topology in order to achieve scalability. Real time multiplayer game 
bring the problem of having low latency since it would be pointless 
without it thus the topology must also cover this part.


\section*{Motivation}
As mentioned in the problem formulation, peer-to-peer real time 
communication using the web is not something that 
has been solved on a higher level (i.e. having scalability and 
low latency). From a economical view point this 
helps the game creator to not spend money on a expensive server 
that handles multiple connections. From a scientific 
point of view, as more and more application is integrated into 
the web browsers due to its broad availability to almost all 
platforms we need better ways for real-time communication using 
only the web browser as our toolbox as this solution can easily 
be extended to not only real-time multiplayer games. 

\section*{Objectives}

\begin{tabular} {|p{1.2cm}|p{11.6cm}|} \hline
\textbf{01} & Selecting some peer-to-peer network topologies that aims for low latency and scalability. \\ \hline 
\textbf{02} & Defining requirements in order to have something to compare against (low latency, high scalability etc). \\ \hline 
\textbf{03} & Compare the selected peer-to-peer network topology in order to lower the amount of topologies. \\ \hline
\textbf{04} & Experimenting with the selected topologies and their corresponding settings to lower the list even more. \\ \hline 
\textbf{05} & Setting up a test/simulation in order to test webRTC topology. \\ \hline 
\textbf{06} & Implement and experiment using the test and webRTC to pick the final topology. \\ \hline 
\textbf{07} & Finalize \& optimize the final implementation. \\ \hline
\end{tabular} 

\section*{Method}
To select the different network topologies a systematic literature review 
(SLR) must first take place to gather the background knowledge. Same 
applies for background knowledge of the webRTC API. Selecting the 
topologies requires a comparative method between the many topologies as a 
result of the SLR in order to lower the list. 

\noindent Running experiments on the topologies with their own settings results of 
a small topology list that will be implemented using webRTC techniques. 
Further experiments and implementation will then be conducted thus resulting
in a final implementation using a specific optimized topology. The final 
implementation will be verified and validated to make sure it runs and 
meet the defined requirements. 


\section*{Time plan}
\begin{tabular} {|p{2.6cm}|p{10.2cm}|} \hline
\textbf{Date} & \textbf{Milestone} \\ \hline
2019-02-15 & Systematic Literature Review for background knowledge (SLR) + selecting potential topologies \\ \hline
2019-02-19 & Degree project plan finished \\ \hline
2019-03-08 & Reading and gather the information generated from the SLR \\ \hline
2019-03-17 & Writing the peer-to-peer background part based on the selected topologies \\ \hline
2019-03-20 & Exploring the webRTC API by both reading the specs and testing with simple implementations  \\ \hline
2019-03-22 & Finishing the background with webRTC \\ \hline
2019-03-23 & Defining the requirements for the topology \\ \hline
2019-03-25 & Writing the motivation and aim (re-using this document with some retouching) \\ \hline
2019-03-27 & Starting with the (method?) and result \\ \hline
2019-04-05 & Experiment with the different topologies to achieve low latency and scalability \\ \hline
2019-04-15 & Implement the most suitable topology into web environment using the webRTC API \\ \hline
2019-04-20 & Verifying and validating the implementation  \\ \hline
2019-04-22 & Discussing the results based on the evaluation  \\ \hline
2019-05-01 & Finalizing the thesis  \\ \hline
\end{tabular}

\pagebreak
\bibliographystyle{IEEEtran}
\bibliography{./references}

\end{document}}